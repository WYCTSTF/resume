% !TEX TS-program = xelatex
% !TEX encoding = UTF-8 Unicode
% !Mode:: "TeX:UTF-8"

\documentclass{resume}
\usepackage{zh_CN-Adobefonts_external} % Simplified Chinese Support using external fonts (./fonts/zh_CN-Adobe/)
% \usepackage{NotoSansSC_external}
% \usepackage{NotoSerifCJKsc_external}
% \usepackage{zh_CN-Adobefonts_internal} % Simplified Chinese Support using system fonts
\usepackage{linespacing_fix} % disable extra space before next section
\usepackage{cite}

\begin{document}
\pagenumbering{gobble} % suppress displaying page number

\name{沈宇昊}

\basicInfo{
  \email{996098413@qq.com} \textperiodcentered\ 
  \phone{(+86) 159-671-51885} \textperiodcentered\ 
  %\linkedin[billryan8]{https://www.linkedin.com/in/billryan8}}
}
 
\section{\faGraduationCap\  教育背景}
% \datedsubsection{\textbf{上海交通大学}, 上海}{2013 -- 至今}
% \textit{在读硕士研究生}\ 信息与通信工程, 预计 2016 年 3 月毕业
\datedsubsection{\textbf{兰州理工大学}, 兰州, 甘肃}{2021 -- 2025}
\textit{本科在读}\ 软件工程

\section{\faUsers\ 项目经历}

\datedsubsection{\textbf{鲲鹏数学库ibm精度库的45个接口生态完善 第三期}}{2023年7月 -- 2023年12月}
\role{C, Python, X86 asm}{校企合作项目,项目核心成员}
\begin{onehalfspacing}
基于鲲鹏920平台,完善libm精度库基础数学函数接口,每个基础数学函数接口精度对标x86 imf2021版本保持100\%一致, https://www.hikunpeng.com/developer/ecology\_remit/D00295
\begin{itemize}
  \item 编写/校验long double精度计算、转换、对比的位运算方法
  \item 整理IDA pro远程调试、gdb动态调试方法,x86相关指令集SSE,跟进项目进度定期汇报。
  \item 通过IDA pro反编译,反汇编结合gdb动态调试,逆向3个 X86下Intel ICC库函数(long double)至arm下华为BiSheng编译器,通过华为精度测试和代码规范评审。
\end{itemize}
\end{onehalfspacing}

% \datedsubsection{\textbf{\LaTeX\ 简历模板}}{2015 年5月 -- 至今}
% \role{\LaTeX, Python}{个人项目}
% \begin{onehalfspacing}
% 优雅的 \LaTeX\ 简历模板, https://github.com/billryan/resume
% \begin{itemize}
%   \item 容易定制和扩展
%   \item 完善的 Unicode 字体支持,使用 \XeLaTeX\ 编译
%   \item 支持 FontAwesome 4.5.0
% \end{itemize}
% \end{onehalfspacing}

% Reference Test
%\datedsubsection{\textbf{Paper Title\cite{zaharia2012resilient}}}{May. 2015}
%An xxx optimized for xxx\cite{verma2015large}
%\begin{itemize}
%  \item main contribution
%\end{itemize}

\section{\faCogs\ 个人技能}
% increase linespacing [parsep=0.5ex]
\begin{itemize}[parsep=0.5ex]
  \item 具备扎实的计算机基础,包括但不限于数据结构与算法,操作系统,计算机网络。
  \item 熟悉C++ STL相关语法,C++11大部分特性。
  \item 熟悉Linux及Linux下开发工具链的环境配置、使用,能编写Python、Shell脚本。
  \item 了解MySQL、PostgreSQL、NoSQL等数据库的装置、权限装备、数据迁移等。
  \item 熟练使用Nginx、Tomcat等Web服务器,负载均衡、反向代理的功能。
  \item 了解HTML/CSS/JS前端三件套,使用过Vue+SpringBoot框架。
  \item 熟练使用Git,熟悉团队协作流程,有开源社区交流经验。
\end{itemize}

\section{\faHeartO\ 获奖情况}
\datedline{校级三好学生}{2021-2022}
\datedline{校级一等奖学金}{2021-2022}
\datedline{蓝桥杯\ 国家级\ 二等奖}{2022}
\datedline{RAICOM编程赛道\ 国家级\ 二等奖}{2022}
\datedline{团体程序设计天梯赛\ 国家级\ 个人三等奖}{2023}
\datedline{RAICOM编程赛道\ 国家级\ 一等奖}{2024}
\datedline{蓝桥杯\ 国家级\ 二等奖}{2024}
\datedline{ICPC全国邀请赛(陕西)\ 铜奖}{2024}

\section{\faInfo\ 其他}
% increase linespacing [parsep=0.5ex]
\begin{itemize}[parsep=0.5ex]
  \item 技术博客: https://blog.syh521.cn
  %\item GitHub: https://github.com/WYCTSTF
  %\item 语言: 英语 - 熟练(TOEFL xxx)
\end{itemize}

%% Reference
%\newpage
%\bibliographystyle{IEEETran}
%\bibliography{mycite}
\end{document}
